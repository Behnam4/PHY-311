\documentclass[fleqn]{article}
\oddsidemargin 0.0in
\textwidth 6.0in
\thispagestyle{empty}
\usepackage{import}
\usepackage{amsmath}
\usepackage{graphicx}
\usepackage{flexisym}
\usepackage{calligra}
\usepackage{amssymb}
\usepackage{bigints} 
\usepackage[english]{babel}
\usepackage[utf8x]{inputenc}
\usepackage{float}
\usepackage[colorinlistoftodos]{todonotes}


\DeclareMathAlphabet{\mathcalligra}{T1}{calligra}{m}{n}
\DeclareFontShape{T1}{calligra}{m}{n}{<->s*[2.2]callig15}{}
\newcommand{\scriptr}{\mathcalligra{r}\,}
\newcommand{\boldscriptr}{\pmb{\mathcalligra{r}}\,}

\definecolor{hwColor}{HTML}{AD53BA}

\begin{document}

  \begin{titlepage}

    \newcommand{\HRule}{\rule{\linewidth}{0.5mm}}

    \center

    \begin{center}
      \includegraphics[height=11cm, width=11cm]{asu.png}
    \end{center}

    \vline

    \textsc{\LARGE Classical Parts/Field/Matter II}\\[1.5cm]

    \HRule \\[0.5cm]
    { \huge \bfseries Problem Set 2}\\[0.4cm] 
    \HRule \\[1.0cm]

    \textbf{Behnam Amiri}

    \bigbreak

    \textbf{Prof: Maulik Parikh}

    \bigbreak

    \textbf{{\large \today}\\[2cm]}

    \vfill

  \end{titlepage}

  \begin{enumerate}
    \item Does the function $f(x,y)=2(x^2+y^2)$ satisfy the two-dimensional Laplace’s equation? Check that the function 
    $g(x, y)=x^2-y^2$ does. Sketch the latter function, calculate the gradient at the points $(0, 1), ~ (1, 0), ~ (0, -1),$ 
    and $(-1, 0)$  and indicate by little arrows the directions in which these gradients point.
    \\
    Within the square bounded by the lines $x=\pm 1$ and $y=\pm 1$, where is $g$
    maximum and where is it a minimum?
    \\
    What do you notice about the location of these points that is characteristic of solutions to Laplace’s equation?

    \item A charge $q$ is placed at the origin, exactly between two parallel, conducting infinite $x−y$ planes, which are separated by a 
    distance 2d in the $z$-direction (so the charge is a distance $d$ away from each plane). Suppose we are interested in the region 
    between the conducting planes, $−d < z < d$. Determine the charges and locations of the image charges
    that would allow us to do away with the conducting planes.


    \item We want to design a spherical capacitor with an outer conducting shell of given radius a and an inner conducting sphere of 
    unknown radius $b$. Suppose we want our capacitor to store the maximum possible energy subject to the constraint that the electric 
    field at the surface of the inner sphere not exceed $E_0$. Find the radius $b$ for the inner sphere as well as the total amount of 
    electrical energy stored in the capacitor.


    \item Is the following argument correct? Explain in detail. A point charge $q$ is located at an off-center position inside an 
    uncharged conducting spherical shell. We know the potential must be constant on the inner surface of the shell. Therefore, by the 
    uniqueness theorem, the potential is constant in the hollow interior as well. Hence the electric field inside is zero and the 
    charge therefore experiences no force


    \item A point charge $−2q$ is placed at the origin and a point charge $3q$ is
    place on the $z$-axis at $(0, 0, b$). Calculate
    \begin{enumerate}
      \item The monopole moment


      \item The dipole moment


      \item And the approximate electric potential at large r to order $(1/r2)$
      in spherical coordinates
      
    \end{enumerate}



    \item A spherical shell has radius $R$ and uniform surface charge density $\sigma$. Find the electric potential 
    at a point on the surface
    \begin{enumerate}
      \item By using the symmetry of the problem
      
      \item By brute-force integration using $\rho=\sigma \delta(r-R)$ where $\delta$ is the
      radial Dirac-delta function obeying $\bigints\limits_{0}^{\infty} \delta(r-R)dr=1$
    \end{enumerate}



    \item An infinite hollow rectangular pipe runs along the $x$-axis from $-\infty < x < \infty$. Suppose that three sides of the pipe 
    (at $y=0$, $z=0$, and $y=a$) are made of metal. If the fourth side (at $z=b$) is kept at the constant potential $V_0$, find the 
    electric potential inside the pipe.


    \item A conducting hollow shell of radius $R$ has charge $Q$. We want to know why the shell doesn’t just spontaneously discharge 
    by having its charge go off into the space around it. (For this problem, we’ll think in terms of classical electrostatics, 
    ignoring the atomic structure of the conductor.) Suppose we try to peel off a charge q. Charge conservation
    says that $Q−q$ remains on the shell. Calculate the potential energy (not electric potential) for the removed charge as a 
    function of distance $r$ from the center of the shell (for $r > R$). Be sure to take into account both the remaining charge on 
    the shell and the surface charge induced on it by the removed charge q (see Griffiths Example 3.2 for help with
    the corresponding image charge). Show that there is an energy barrier preventing the shell from spontaneously discharging.
  


  \end{enumerate}

\end{document}