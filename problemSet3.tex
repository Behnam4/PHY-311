\documentclass[fleqn]{article}
\oddsidemargin 0.0in
\textwidth 6.0in
\thispagestyle{empty}
\usepackage{import}
\usepackage{amsmath}
\usepackage{graphicx}
\usepackage{flexisym}
\usepackage{calligra}
\usepackage{amssymb}
\usepackage{bigints} 
\usepackage[english]{babel}
\usepackage[utf8x]{inputenc}
\usepackage{float}
\usepackage[colorinlistoftodos]{todonotes}


\DeclareMathAlphabet{\mathcalligra}{T1}{calligra}{m}{n}
\DeclareFontShape{T1}{calligra}{m}{n}{<->s*[2.2]callig15}{}
\newcommand{\scriptr}{\mathcalligra{r}\,}
\newcommand{\boldscriptr}{\pmb{\mathcalligra{r}}\,}

\definecolor{hwColor}{HTML}{531C55}

\begin{document}

  \begin{titlepage}

    \newcommand{\HRule}{\rule{\linewidth}{0.5mm}}

    \center

    \begin{center}
      \includegraphics[height=11cm, width=11cm]{asu.png}
    \end{center}

    \vline

    \textsc{\LARGE Classical Parts/Field/Matter II}\\[1.5cm]

    \HRule \\[0.5cm]
    { \huge \bfseries Problem Set 3}\\[0.4cm] 
    \HRule \\[1.0cm]

    \textbf{Behnam Amiri}

    \bigbreak

    \textbf{Prof: Maulik Parikh}

    \bigbreak

    \textbf{{\large \today}\\[2cm]}

    \vfill

  \end{titlepage}

  \begin{enumerate}
    \item The electric dipole moment of a single water molecule is about $6 \times 10^{-30}$ coulomb-meters. Imagine that all
    the molecular dipoles in a cup of water could be made to point down. Calculate the magnitude of the resulting surface charge 
    density at the upper surface of the water. How many electrons per square centimeter does that correspond to?

      \textcolor{hwColor}{
        \\
        \includegraphics[height=5cm, width=6cm]{1.JPG}
        \\
        The dipole moment arises because oxygen is more electronegative than hydrogen; the oxygen pulls in the shared electrons 
        and increases the electron density around itself.The asymmetry of the water molecule leads to a dipole moment in the symmetry 
        plane pointed toward the more positive hydrogen atoms. The measured magnitude of this dipole moment is
        $$
          p=6 \times 10^{-30} ~ C.m 
        $$
        By consulting the Periodic Table of elements we get that water has a molecular weight of $18$ grams per mole. 
        We can convert moles to number of molecules by multiplying by Avogadro’s number.
        \\
        \\
        $
          n=\dfrac{1}{18} \times 6.02 \times 10^{23} \approx 3.33 \times 10^{22} ~~ \text{molecules per $cm^{-3}$}
        $
        \\
        \\
        We are told that the dipoles are all point down, therefore the polarization density is:
        \\
        \\
        $
          P=np=\left(
            3.33 \times 10^{22} ~~ \text{molecules per $cm^{-3}$}
          \right) \times 
          \left(
            6 \times 10^{-30} ~ C.m 
          \right)
          \\
          \\
          =\left(
            3.33 \times 10^{22} ~~ \text{molecules per $cm^{-3}$}
          \right) \times 
          \left(
            6 \times 10^{-28} ~ C.cm 
          \right)
          \\
          \\
          \\
          \therefore ~~~ P=1.998 \times 10^{-5} ~~~ C/cm^2 ~~~~ \checkmark
        $
        \\
        \\
        What we have now is the surface charge density $\sigma$. The number of electrons per square centimeter is:
        \\
        \\
        $
          \dfrac{\sigma}{e}=\dfrac{1.998 \times 10^{-5} ~~~ C/cm^2}{1.60 \times 10^{-19} ~~~ C}
          \\
          \\
          \\
          \therefore ~~~ \dfrac{\sigma}{e}=1.24875 \times 10^{14} ~~~ electrons/cm^{2} ~~~~ \checkmark
        $
        \\
      }

    \item An electric dipole $\overrightarrow{p}$ is at the origin, at one corner of an equilateral triangle of side a, pointing 
    along one side (which you can take to be along the $\hat{+z}$ direction). A positive test charge q is initially at the corner at 
    the other end of that side (i.e. at $z=a$). Calculate the work done in moving the charge to the third corner of the triangle.

      \textcolor{hwColor}{
        \\
        \includegraphics[height=5cm, width=7cm]{2.JPG}
        \\
        An electric dipole is a charge distribution that is made up of two equal and opposite point charges.
        \\
        \includegraphics[height=5cm, width=7cm]{3.JPG}
        \\
        $
          V=V_++V_-=\dfrac{1}{4 \pi \epsilon_0} \dfrac{q}{r_1}+\dfrac{1}{4 \pi \epsilon_0} \dfrac{-q}{r_2}
          =\dfrac{q}{4 \pi \epsilon_0} \left[
            \dfrac{1}{r_1}-\dfrac{1}{r_2}
          \right]
          \\
          \\
          \\
          \begin{cases}
            cos(\theta)=\dfrac{OC}{a}
            \\
            \\
            cos(\theta)=\dfrac{OD}{a}
          \end{cases} \Longrightarrow \begin{cases}
            r_1 \approx r-a cos(\theta)
            \\
            \\
            r_2 \approx r+a cos(\theta)
          \end{cases}
          \\
          \\
          \\
          V=\dfrac{q}{4 \pi \epsilon_0} \left[
            \dfrac{1}{r-a cos(\theta)}-\dfrac{1}{r+a cos(\theta)}
          \right]
          =\dfrac{q}{4 \pi \epsilon_0} \left[
            \dfrac{r+a cos(\theta)-\left(r-a cos(\theta)\right)}{r^2-a^2 cos^2(\theta)}
          \right]
          =\dfrac{q}{4 \pi \epsilon_0} \left[
            \dfrac{2a cos(\theta)}{r^2-a^2 cos^2(\theta)}
          \right]
        $
        \\
        \\
        From the textbook (page 155) we learned that for a physical dipole we have $p=qd$ where d is the vector from the negative
        charge to the positive, therefore
        \\
        \\
        $
          V=\dfrac{1}{4 \pi \epsilon_0} \left[
            \dfrac{p cos(\theta)}{r^2-a^2 cos^2(\theta)}
          \right]
          \\
          \\
          \\
          \therefore ~~~ V=\dfrac{1}{4 \pi \epsilon_0} \left[
            \dfrac{p cos(\theta)}{r^2}
          \right] ~~~~ r>>a
        $
        \\
        \\
        Since $p.\hat{r}=p cos(\theta)$, where $\hat{r}$ is the unit vector along the OP then electric potential of dipole is
        \\
        \\
        $
          \therefore ~~~ V=\dfrac{1}{4 \pi \epsilon_0} \dfrac{p.\hat{r}}{r^2} ~~~~ \checkmark
        $
        \\
        \\
        Voltage is the pressure from an electrical circuit's power source that pushes charged electrons (current) through 
        a conducting loop. This means that one coulomb of charge will gain one joule of potential energy when it is moved 
        between two locations where the electric potential difference is one volt. In a static electric field, the work 
        required to move per unit of charge between two points is known as voltage. Mathematically, the voltage can be expressed 
        as: (All three angles of an equilateral triangle are $\dfrac{\pi}{3}$)
        \\
        \\
        $
          \Delta V=\dfrac{W}{Q}  \Longrightarrow W=\Delta V \times Q ~~~~ \checkmark
          \\
          \\
          \\
          W=q \times \left[
          \dfrac{1}{4 \pi \epsilon_0} \dfrac{p ~ cos(\theta_f)}{r^2}
          -\dfrac{1}{4 \pi \epsilon_0} \dfrac{p ~ cos(\theta_i)}{r^2}
          \right]
          =q \times \left[
            \dfrac{1}{4 \pi \epsilon_0} \dfrac{p ~ cos(\pi/3)}{r^2}
            -\dfrac{1}{4 \pi \epsilon_0} \dfrac{p ~ cos(0)}{r^2}
          \right]
          \\
          \\
          \\
          =q \times \left[
            \dfrac{1}{4 \pi \epsilon_0} \dfrac{p \times \dfrac{1}{2}}{r^2}
            -\dfrac{1}{4 \pi \epsilon_0} \dfrac{p}{r^2}
          \right]
          =\dfrac{q}{4 \pi \epsilon_0} \left[
            \dfrac{p}{2r^2}-\dfrac{p}{r^2}
          \right]
          =\dfrac{q}{4 \pi \epsilon_0} \left[\dfrac{p-2p}{2r^2}\right]
          \\
          \\
          \\
          \therefore ~~~ W=-\dfrac{1}{4 \pi \epsilon_0} \dfrac{q p}{2r^2} ~~~~ \checkmark
        $
        \\
        \\
        \emph{Note that for our case $r=a$}.
        \\
        \\
        We know that the direction of an electrical field at a point is the same as the direction of the electrical 
        force acting on a positive test charge at that point. We have the work done as a negative value which means 
        the displacement is opposite to the direction of the Force applied.
        \\ 
      }

    \pagebreak

    \item Consider a parallel-plate capacitor for which the plates are separated by a vertical distance $2d$. Suppose that the 
    space between the plates is filled with two slabs of linear dielectrics, each of thickness $d$. The upper slab has dielectric 
    constant $3$ and the lower slab has dielectric constant $2$. If the free charge density on the upper and lower plate is
    $+\sigma$ and $-\sigma$ respectively, calculate for each slab
    \begin{enumerate}
      \item The electric displacement.

        \textcolor{hwColor}{
          \\
          From page 182 of the textbook we have the following:
          \\
          \\
          $
            \bigoint \overrightarrow{D} . d\overrightarrow{a}=Q_{f_{enc}}
            \\
            \\
            \therefore ~~~ D.A=\sigma_i A \Longrightarrow \begin{cases}
              \overrightarrow{D}_{T}=- \sigma \hat{z}
              \\
              \\
              \overrightarrow{D}_{B}=- \sigma \hat{z}
            \end{cases}
          $
          \\
          \\
          \\
        }

      \item The electric field.

        \textcolor{hwColor}{
          \\
          From page 183 of the textbook we have:
          \\
          \\
          $
            \overrightarrow{E}=\dfrac{1}{\epsilon_0} D, ~~~ \epsilon=\epsilon_0 ~ \epsilon_r
          $
          \\
          For the upper one we have:
          \\
          \\
          $
            \overrightarrow{E}=\dfrac{\sigma}{3 ~ \epsilon_0} ~ \hat{z}
          $
          \\
          \\
          For the lower one we have:
          \\
          \\
          $
            \overrightarrow{E}=-\dfrac{\sigma}{2 ~ \epsilon_0} ~ \hat{z}
          $
          \\
          \\
        }

      \item The polarization.

        \textcolor{hwColor}{
          \\
          On page 186 of the textbook we learned that $\epsilon_r=1+\chi_e=\dfrac{\epsilon}{\epsilon_0}$ and 
          $P=\epsilon_0 ~ \chi_e ~ E$.
          \\
          \\
          For the upper one we have:
          \\
          \\
          $
            P_u=(3-1)\epsilon_0 \left(\dfrac{-\sigma}{3 \epsilon_0}\right) \Longrightarrow P_u=-\dfrac{2 \sigma}{3} \hat{z}
          $
          For the lower one we have:
          \\
          \\
          $
            P_l=(2-1)\epsilon_0 \left(\dfrac{-\sigma}{2 \epsilon_0}\right) \Longrightarrow P_l=-\dfrac{\sigma}{2} \hat{z}
          $
          \\
          \\
        }

      \item The location and value of the bound charges.

        \textcolor{hwColor}{
          \\
          Page 174 of the textbook states that $\rho_b=P.\hat{n}$.
          \\
          \\
          Fopr the upper one we have:
          \\
          \\
          $
            \begin{cases}
              \sigma_{t, u}=\left(\dfrac{-2 \sigma \hat{z}}{3}\right).\left(\hat{z}\right)=-\dfrac{2}{3} \sigma ~~~~ \checkmark
              \\
              \\
              \sigma_{t, l}=\left(\dfrac{-2 \sigma \hat{z}}{3}\right).\left(-\hat{z}\right)=\dfrac{2}{3} \sigma ~~~~ \checkmark
            \end{cases}
          $
          \\
          \\
          For the lower one we have:
          \\
          \\
          $
            \begin{cases}
              \sigma_{b, u}=\left(-\dfrac{\sigma \hat{z}}{2}\right).\left(\hat{z}\right)=-\dfrac{\sigma}{2} ~~~~ \checkmark
              \\
              \\
              \sigma_{b, l}=\left(-\dfrac{\sigma \hat{z}}{2}\right).\left(-\hat{z}\right)=\dfrac{\sigma}{2}  ~~~~ \checkmark
            \end{cases}
          $
          \\
          \\
        }
  
    \end{enumerate}

    \item A particle of charge q and rest mass mis moving with velocity $\overrightarrow{v}$ in a region where there 
    is a uniform magnetic field $\overrightarrow{B}$ perpendicular to $\overrightarrow{v}$. The Lorentz force causes 
    the particle to move in a circular path. Find the time required to complete one revolution. Specifically, consider
    a proton moving at a speed of $10,000 ~ km/s$ (which is still far slower than the speed of light, so ignore relativistic 
    effects) through a perpendicular interstellar magnetic field of strength $10^{-9} T$. Find the orbital time for the proton.

      \textcolor{hwColor}{
        \\
        For a particle in uniform circular motion we have:
        \\
        \\
        $
          \overrightarrow{F_B}=q \left(
            \overrightarrow{v} \times \overrightarrow{B}
          \right)
        $
        \\
        \\
        Since the field is perpendicular to $\overrightarrow{v}$ we have:
        \\
        \\
        $
          \overrightarrow{F_B}=q v B sin(\dfrac{\pi}{2}) \Longrightarrow \overrightarrow{F_B}=q v B
        $
        \\
        \\
        We know that a centripetal force $F_C$ is a force that makes a body follow a curved path. Since we are in a circular path we can state that:
        \\
        \\
        $
          F_B=F_C=\dfrac{mv^2}{r} \Longrightarrow q v B=\dfrac{mv^2}{r}
          \\
          \\
          \\
          \therefore ~~~~  r=\dfrac{mv}{qB} ~~~~ \checkmark
        $
        \\
        \\
        $r$ is the radius of curvature of the path of a charged particle with mass $m$ and charge $q$, moving at a speed $v$ that is 
        perpendicular to a magnetic field of strength $B$. We can define the time for the charged particle that goes around the circular path
        as the period which is equal to the distance traveled divided by the speed.
        \\
        \\
        $
          T=\dfrac{2 \pi r}{v}=\dfrac{2 \pi \left(\dfrac{mv}{qB}\right)}{v}
          \\
          \\
          \\
          \therefore ~~~ T=\dfrac{2 \pi m}{q B} ~~~~ \checkmark
          \\
        $
        \\
        We are asked to find the orbital time for a proton moving at a speed of 10000 $km/s$.
        \\
        \\
        $
          \begin{cases}
            m_{proton}=1.6726219 × 10^{-27} ~ kg
            \\
            \\
            q_{proton}=1.602176634×10^{−19} ~ C
            \\
            \\
            B=10^{-9} ~ T
          \end{cases}
          \\
          \\
          \\
          T=\dfrac{2 \pi m}{q B}=\dfrac{\left(2 \pi\right) ~ \left(1.6726219 × 10^{-27} ~ kg\right)}{\left(1.602176634×10^{−19} ~ C\right) ~ \left(10^{-9} ~ T\right)}
          \\
          \\
          \\
          \therefore ~~~ T=65.5944 ~ s
        $
        \\ 
      }

    \pagebreak

    \item A cube of side $L$, centered at the origin, is made of a dielectric with a polarization $\overrightarrow{P}=k \overrightarrow{r}$.
    Find the bound charges and check that they add up to zero.

      \textcolor{hwColor}{
        \\
        Any kind of matter is full of positive and negative electric charges. In a dielectric, these charges are bound. They cannot 
        move separately from each other through any macroscopic distance. Hence when an electric field is applied there is no net electric
        current. However, the field does push the positive charges just a tiny bit in the direction of $E$ while the negative
        charges are pushed in the opposite directions. Consequently, the atoms and the molecules
        comprising the dielectric acquire tiny electric dipole moments in the direction of $E$.
        \\
        \includegraphics[height=5cm, width=7cm]{4.JPG}
        \\
        We are told that the polarization is $\overrightarrow{P}=k \overrightarrow{r}$, where 
        $\overrightarrow{r}=x ~ \hat{x}+y ~ \hat{y}+z ~ \hat{z}$. On page 174 of the textbook we have
        \\
        \\
        $
          \rho_b=-\nabla . P=-\left(
            \dfrac{\partial}{\partial x}\hat{x}+\dfrac{\partial}{\partial y}\hat{y}+\dfrac{\partial}{\partial z}\hat{z}
          \right).\left(
            kx ~ \hat{x}+ky ~ \hat{y}+kz ~ \hat{z}
          \right)
          \\
          \\
          \\
          =-\left(
            \dfrac{\partial ~ kx}{\partial x}+\dfrac{\partial ~ ky}{\partial y}+\dfrac{\partial ~ kz}{\partial z}
          \right)
          \\
          \\
          \\
          \therefore ~~~ \rho_b=-3k ~~~~ \checkmark
        $
        \\
        \\
        The total volume charge is $-3kL^3 ~~~~ (A)$
        \\
        \\
        The surface polarization charge density is calculated by $\sigma_b=P.\hat{n}$ where $\hat{n}$ is a normal outward. There are six 
        surfaces for a cube.
        \\
        \\
        $
          \begin{cases}
            P.\hat{x}=\left(k ~ \dfrac{L}{2} ~ \hat{x}\right).\hat{x}=\dfrac{1}{2}kL
            \\
            \\
            P.\left(\hat{-x}\right)=\left(k ~ \dfrac{-L}{2} ~ \hat{x}\right).\left(\hat{-x}\right)=\dfrac{1}{2}kL
            \\
            \\
            P.\hat{y}=\left(k ~ \dfrac{L}{2} ~ \hat{y}\right).\hat{y}=\dfrac{1}{2}kL
            \\
            \\
            P.\left(\hat{-y}\right)=\left(k ~ \dfrac{-L}{2} ~ \hat{y}\right).\left(\hat{-y}\right)=\dfrac{1}{2}kL
            \\
            \\
            P.\hat{z}=\left(k ~ \dfrac{L}{2} ~ \hat{z}\right).\hat{z}=\dfrac{1}{2}kL
            \\
            \\
            P.\left(\hat{-z}\right)=\left(k ~ \dfrac{-L}{2} ~ \hat{z}\right).\left(\hat{-z}\right)=\dfrac{1}{2}kL
          \end{cases}
        $
        \\
        \\
        \\
        And the charge on each face is $q=\dfrac{1}{2}kL^3$ (area). Therefore, total surface charge is
        \\
        \\
        $
          \sum\limits_{n=1}^{6} q_n=\sum\limits_{n=1}^{6} q_n=6 \times \dfrac{1}{2}kL^3
          \\
          \\
          \\
          \therefore ~~~ \sum\limits_{n=1}^{6} q_n=3kL^3 ~~~~ (B)
        $
        \\
        \\
        Now the total is (from (A) and (B))
        \\
        \\
        $
          Q_{total}=3kL^3+(-3kL^3)
          \\
          \\
          \\
          \therefore ~~~ Q_{total}=0 ~~~~ \checkmark
        $
        \\
      }

    \pagebreak
  
    \item Three capacitors have identical area and plate separation. To be definite, let the plate separation direction be 
    the z-direction, and suppose the plates are separated by a distance $d$. Suppose one of the capacitors has only vacuum between the plates. The other two capacitors are
    each half-filled with a linear dielectric material, with dielectric constant $k$. In one of these, the dielectric material is 
    spread across all of one plate but extends only halfway to the other plate (i.e. to $d/2$ in the z-direction). In the other 
    capacitor, the same volume of dielectric material is present but arranged differently, covering only half the plate
    but extending all the way to the other plate. If the vacuum capacitor has a capacitance of $C_0$, find the capacitance of each 
    of the other capacitors.

      \textcolor{hwColor}{
        \\
        Let's start off with the the capacitance of a capacitor formula.
        \\
        \\
        $
          C=\epsilon_r C_{vac} \approx \epsilon_0 ~ \epsilon ~ \dfrac{A}{d}
        $
        \\
        \\
        We are told that we have vacuum between the plates of the capacitors therefore $C_0=\epsilon_0\dfrac{A}{d}$. For the case where 
        the dielectric covers the whole area of the two plates, the capacitance is
        \\
        \\
        $
          C=k ~ C_0
        $. 
        \\
        \\
        When the dielectric covers halfway of one plate and all of the other one (series)
        \\
        \\
        $
          \dfrac{1}{C_{eq}}=\dfrac{d/2}{k ~ \epsilon_0 A}+\dfrac{d/2}{\epsilon_0 A}=\dfrac{1}{2k C_0}+\dfrac{k}{2kC_0}
          \\
          \\
          \\
          \therefore ~~~ C_{eq}=C_0\dfrac{2k}{1+k} ~~~~ \checkmark
        $   
        \\
        \\
        The last case is when the dielectric covers half of one plate but covers the other one.
        \\
        \\
        $
          C_{eq}=k \epsilon_0 \dfrac{0.5 A}{d}+\epsilon_0 \dfrac{0.5 A}{d}=\dfrac{1}{2} k C_0+\dfrac{1}{2} C_0
          \\
          \\
          \\
          \therefore ~~~ C_{eq}=C_0 ~ \dfrac{k+1}{2} ~~~~ \checkmark
          \\
          \\
        $
      }

  \end{enumerate}

\end{document}
